% arara: pdflatex
% arara: biber
% arara: pdflatex
% arara: pdflatex
%%%%%%%%%%%%%%%%%%%%%%%%%%%%%%%%%%%%%%%%%%%%%%%%%%%%%%%%%%%%%%
%% For more info about how to edit this, check AltaCV page: %%
%%                                                          %%
%% https://github.com/liantze/AltaCV                        %%
%%%%%%%%%%%%%%%%%%%%%%%%%%%%%%%%%%%%%%%%%%%%%%%%%%%%%%%%%%%%%%

\documentclass[10pt,a4paper,ragged2e,withhyper]{altacv}
\usepackage{custom}

\addbibresource{publications.bib}

\begin{document}

\name{Victor Santos}
\tagline{Data Engineer | Data Scientist | Climber | Geek}
\photoR{2.5cm}{assets/img/avatar.jpg}
\personalinfo{
  \email{vct.santos@protonmail.com}
  \phone{+55-85-98188-2001}
  \location{Fortaleza, Ceará}
  \homepage{https://vsantos.me/}
  \linkedin{vctsantos}
  \gitlab{padawanphysicist}
  \orcid{0000-0001-6725-2761}
}

\makecvheader
\begin{justify}
  {\small %
    As a Data Engineer at the Fortaleza Planning Institute (IPLANFOR), I worked with a small team to develop and maintain data pipelines, databases, and dashboards to aid the creation and monitoring of public policies in urban planning and management. This provided reliable and actionable data insights for decision makers. I have a strong academic background in Physics, with a PhD in Theoretical and Mathematical Physics from the Federal University of Ceará, where I also taught as a Lecturer for 1.5 years. I have published multiple papers in peer-reviewed journals and presented my research at international conferences. Although my research involves theoretical aspects in Physics, I am passionate about the possible applications of Physics and its methods to real-world problems.%
  }
\end{justify}

\AtBeginEnvironment{itemize}{\small}

\columnratio{0.6}

\begin{paracol}{2}

\cvsection{Professional Experience}

\cvevent{Data Engineer \& Data Scientist}{Fortaleza Planning Institute (IPLANFOR)}{2021 -- 2024}{Fortaleza, Ceará, Brazil}
\begin{itemize}
\item Creation of the \href{https://observatoriodefortaleza.fortaleza.ce.gov.br/dados/}{Data Observatory Platform}, aggregating data from the city and analyzes used in monitoring public policies
\item Assistance in the Situation Room
\item Assistante in the writing of studies and technical notes
\item Participation in the structuring of the City Hall \href{https://www.fortaleza.ce.gov.br/noticias/fortaleza-cidade-inteligente-prefeitura-anuncia-central-integrada-e-politica-publica-de-videomonitoramento}{video surveillance central}
\end{itemize}

\textbf{Main technologies}: R (Shiny), Python (Pandas, Prefect), PostgreSQL, Docker, OpenShift

\divider

\cvevent{Data Scientist}{Fundação Cearense de Apoio ao Desenvolvimento Científico e Tecnológico (FUNCAP)}{2018 -- 2022}{Fortaleza, Ceará}

\begin{itemize}
\item Analysis and evaluation of scientific research developed in the state of Ceará, establishing an overview of the main areas funded by the government and a profile of the researchers and publications within the national and international scenario, through collecting and analyzing of data from national and international publishing databases, such as the \href{https://lattes.cnpq.br/}{Lattes platform}, \href{https://www.scielo.br}{Scielo}, \href{https://scholar.google.com.br/}{Google Scholar} and \href{http://montenegro.funcap.ce.gov.br/sugba/}{Montenegro platform}.
\end{itemize}
\textbf{Main technologies}: R, Ruby, Docker, PostgreSQL, BaseX, ArangoDB

\divider

\cvevent{Assistant Professor}{Universidade Federal do Ceará}{2017 -- 2018}{Fortaleza, Ceará}

\begin{itemize}
\item Lecturing Basic Physics courses for Engineering, Physics and Geology
\end{itemize}
\textbf{Main technologies}: Python (for simulations), LaTeX

\divider

\cvevent{Postdoc researcher}{Universidade Federal do Ceará}{2015 -- 2016}{Fortaleza, Ceará}

\begin{itemize}
\item Research on quantum aspects of gravitational theories
\item Mentoring graduate students
\end{itemize}
\textbf{Main technologies}: Python, LaTeX

%% Switch to the right column. This will now automatically move to the second
%% page if the content is too long.
\switchcolumn

\cvsection{Hard skills}
\cvtag{\href{https://www.python.org/}{Python}}
\cvtag{\href{https://www.r-project.org/}{R}}
\cvtag{\href{https://cplusplus.com/}{C/C++}}
\cvtag{\href{https://www.ruby-lang.org/}{Ruby}}
\cvtag{\href{https://fortran-lang.org/}{Fortran}}
\cvtag{\href{https://developer.mozilla.org/pt-BR/docs/Web/JavaScript}{JavaScript}}
\cvtag{\href{https://lisp-lang.org/}{Common Lisp}}
\cvtag{HTML/CSS}


\cvtag{\href{https://www.docker.com/}{Docker}}
\cvtag{\href{https://www.postgresql.org/}{PostgreSQL}}
\cvtag{\href{https://basex.org/}{BaseX}}
\cvtag{\href{https://www.arangodb.com/}{ArangoDB}}
\cvtag{Data Mining}

\cvsection{Strengths}

\cvtag{Flexibility}
\cvtag{Hard-working}
\cvtag{Patience}\\
\cvtag{Motivation}
\cvtag{Team Work}

\cvtag{Analytical/Mathematical Thinking} 

\cvsection{Languages}

\cvskill{English}{4}

\cvskill{Português (Native)}{5}

\cvsection{Formation}

\cvevent{PhD in Physics}{Universidade Federal do Ceará}{}{}

\divider

\cvevent{B.Sc. in Physics}{Universidade Federal do Ceará}{}{}

\cvsection{Last Publications}

\nocite{*}

\printbibliography[heading=pubtype,title={\printinfo{\faFile*[regular]}{Papers}}, type=article]

\end{paracol}

\end{document}
